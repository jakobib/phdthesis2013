\section{Philosophy}
\label{sec:philosophy}

Philosophy is the principal field of research, which all other scientific
disciplines originate from. It typically adresses questions, that are ignored 
by other fields, including itself and its own methods. Philosophy 
critically examines beliefs and finds problems, that are difficult to answer
within the limited scope of one domain. Such examinations are not always
welcome, or regarded as somehow interesting but essentially irrelevant. 
Philosophy often deals with `blind spots' like hidden assumptions and
foundations. Such blind spots also exist in library and information science 
and in computer science in regard to the description of digital documents.
This section summarizes some existing philosophical approaches and questions
about documents and data, that may unhide these hidden assumptions.

\subsection{Philosophy of Information} 

Connections between philosophy and \tacro{library and information science}{LIS}
with emphasis on the concept of information have been drawn by
\textcite{Floridi2002} and by \textcite{Capurro2003}.\footnote{The intellectual
exchange between this philosophers, however, is suprisingly low. I found no
explicit reference from Floridi to Capurro, one rejection of Floridis position
by \textcite{Capurro2008}, an overview by \textcite{Compton2006} and a
comparision of both in Persian \cite{Khandan2009}.} Floridi argues that
\tacro{philosophy of information}{PI} is the philosophical discipline that can
best provide the conceptual foundations for \acro{LIS}. The main question of
\acro{PI}, as coined by \textcite{Floridi2002w}, is ``What is information?''.
How does this question relate to the description of documents and data?
\textcite{Floridi2002} defines \acro{LIS} as applied \acro{PI}, which in his
words is ``concerned with documents, their life cycles and the procedures,
techniques and devices by which these are implemented, managed and regulated.''
By this it ``does not cover all \acro{PI}’s ground'' but ``information [\ldots]
in the weaker and more specific sense of recorded data or \emph{documents}''.
This definition ignores objects of \acro{LIS}, that are not primarily focused
on documents, such as \term{information literacy}. \textcite{Cornelius2004} in
his response to \textcite{Floridi2002} argues that \acro{LIS} ``has
reconstructed itself away from an overwhelming concern with information
materials (documents) and their organizational systems to an equal concern with
the behavior of individual people who use libraries, and documents.'' Documents
alone do not make a library, if they are not put in a cultural and social
context.  Unless you share the positivist view, in which documents hold an
objective and stable meaning, you cannot separate documents from their
description.

%Floridi locates \acro{LIS} at a more specific sub-discipline of \acro{PI},
%without further investigation. 

Current research in \acro{PI}, as summarized in \textcite{Floridi2009}, does
not deeply deal with concepts like documents and recorded data. More specific
questions like `'What is a document?'', ``What is a data?'', and ``What does it
mean to be recorded?'' better capture the philosophical foundations of digital
documents. But they are rarely asked in \acro{PI}, although they refer to the
fundamentals of this discipline: In particular, the \Tacro{General Definition
of Information}{GDI}, which defines information as is well-formed, meaningful
data, has widely been accepted.\footnote{\textcite{Floridi2005} adds that
(semantic) information must also be veridical, but we can ignore this aspect.}
Data however, is ``intuitively described as uninterpreted differences (symbols
or signals).'' \cite{Floridi2009}. The terms data and symbols are even used
synonym when Floridi refers to the  \Tacro{Symbol Grounding Problem}{SGP} as
``data grounding problem''. The data/\term{symbol grounding problem} is one of
the open problems in \term{philosophy of information}
\cite{Harnad2007,Taddeo2005}.\footnote{ We will ignore positions like those of
\textcite{Steels2007} and \textcite{Taddeo2007}, which argue, that the
\term{symbol grounding problem} had been solved.} In short \acro{SGP} asks, how
symbols aquire their meaning. The question was raised by
\Person[Stevan]{Harnad}, inspired by \person[John R.]{Searle}'s \Term{Chinese
Room Argument}. The latter shows that manipulation of formal symbols (as
defined in terms of \term{formal languages}) is not by itself constitutive of,
nor sufficient for, thinking. So how can symbols inside an autonomous system be
connected to their referents in the external world, without the mediation of an
external interpreter? I disagree with the view that ``data constituting
information can be meaningful independent of an informee'' \cite[p.
22]{Floridi2010}. \acro{SGP} cannot be solved, because symbols always require
some mediation. Data are not ``uninterpreted differences (symbols or
signals)'', but you already require interpretation to draw distinctions, which
are needed to constitute symbols. Apparently we need a deeper philosophical
look at data, not simply derived from concepts like information and meaning.

\subsection{Philosophy of Data}
\label{sec:philofdata}

The concept of \Term{data} is rarely studied as main topic of philosophical
investigation, but mostly mixed with the discussion of information, knowledge, 
and meaning. Even in disciplines that use data as a central concept, there is 
no single, commonly accepted definition of data. \textcite{Gray2003} has shown 
for the early field of information systems, that until the 1960s authors made
no clear distinction between data and information. The same can also be 
observed in othere areas and in more recent literature.

Data is often assumed as `something given', as signified by the latin root
\emph{datum}. The latin term originates from a translation of \Person{Euklid}'s
work $\Delta\epsilon\delta o\epsilon\nu\alpha$ from the 4th century BC, in
which he deals with geometric problems. Later the term was mainly used to
discuss epistomological question in philosophy of perception --- for instance
\Person[Bertrand]{Russel}'s concept of \Term{sense-data} --- and in philosophy
of science. In order to assess existing philosophical understandings of data,
\textcite{BallsunStanton2010,BallsunStanton2012} used methods of experimental
philosophy.\footnote{Experimental philosophy supplements analytical
philosophy with empirical methods like surveys to discover how people
ordinarily think about concepts.} He found three general types of data: data as
communicated bits, data as hard numbers of objective facts, and data as
recorded but subjective observations. Independent of this conceptions, some
people distinguish raw data, and derived data, which origins from raw data by
automatic calculations.  The philosophy of \Term{data as bits} considers
computers as only arbiter of data, and of only data. To transform data into
information and knowledge, humans need to analyse and interprete the input and
output of computed actions.  The philosophy of \Term{data as hard numbers}
views data as product of objective, reproducible, and unambiguous measurements.
This data requires a set of precisely understood metadata, which itself does
not count as data. Data as hard numbers are mostly used in scientific contexts,
where statistical data analysis is the main method to derive data from other
data. The philosophy of \Term{data as recorded observations} understands data
from an engineering perspective, as recorded product of observations.
Everything produces data and our knowledge is needed to select relevant
instances.  Data can be turned into information and knowledge by
contextualization against other data, information, and knowledge in a
hierarchical process or in feedback cycles. A summary of these different
philosophies is given in table~\ref{tab:dataphilosophies}. In the following we
will ignore the positivist view of data as hard numbers --- most digitital
docments do not simply origin from subjective or objective observations, but
from designed creations.

\begin{table}
\centering
\begin{tabular}{|l|l|}
  \hline
  philosophy of data              & data as \\
  \hline
  data as hard numbers            & objective observation  \\
  data as recorded observations   & subjective observation \\
  data as bits                    & creation instead of observation \\
  \hline
\end{tabular}
\caption{General philosophies of data, based on Ballsun-Stanton}
\label{tab:dataphilosophies}
\end{table}

% data as recognized observations: metadata as data used to contextualize other data

In philosophy of information the concept of data is only covered briefly to
define information. A serious treatment of the term `\term{metada}' in
particular does not exist. \textcite{Floridi2005,Floridi2010} refers to the
\Term{diaphoric definition of data}, and gives a simple classification with
five, non mutually exclusive types of data (primary data, secondary data,
metadata, operational data, and derivative data) In his definition data is 

\begin{quotation}%
$x$ being distinct from $y$, where $x$ and $y$ are two uninterpreted variables
and the relation of `being distinct`, as well as the domain, are left open
to further interpretation.\\
\quotationsource \textcite[p. 23]{Floridi2010}
\end{quotation}

\noindent
More formally, a set of data can be described as $\set[(\neq,x,y)]{x \neq y}$,
where `$\neq$' denotes a lack of uniformity between $x$ and $y$. This definition
is linked to the problem of \term{identity}, which has a much longer
philosophical history. The general five types of data are vaguely defined with
direct reference to the information that they convey: \Term{primary data} is 
the general form of data, as stored in a technical system.
\Term{Secondary data} is constituted by the absence of primary data. Floridi 
argues, that the absence of information is also information. The conclusion,
that the absence of (primary) data should be secondary data, is less clear.
\Term{Metadata}, to fully quote his words, ``are indications about the nature
of some other (usually primary) data. They describe properties such as location,
format, updating, availability, usage restrictions, and so forth.''.
\Term{Operational data} regard the operations of data systems. In contrast
to metadata, which describe properties of data, operational data describe
properties of systems that processes data. Finally \Term{derivative data}
are data that can be extracted from other data whenever the latter are
used as clues about other things than directly adressed by the data 
themselves. In other words, operational data is data that conveys different
information, if put in another context than originally intended.

Floridi further describes three applications of the diaphoric definition of 
data: pure data or proto-epistemic data `de re', that is a lack of uniformity 
before any interpretation or cognitive processing; data `de signo' as lack of 
uniformity between (the perception of) physical states or signals; and data 
`de dicto' as lack of uniformity between symbols, for example between the 
letters A and B. Without further critique of this typology, we can limit 
analysis on data `de dicto', namely the bits, that each digital document is
made of. A summary of Floridi's types of data is given in 
table~\ref{tab:floridisdatatypes}.

\begin{table}
\begin{minipage}[t]{0.45\linewidth}
\renewcommand{\arraystretch}{1.3}
\begin{tabular}{|rl|}
\hline
\multicolumn{2}{|c|}{primary data} \\
secondary data\hspace{6mm} & \hspace{4mm}derivative data \\
operational data & metadata \\
\hline
\end{tabular}
\renewcommand{\arraystretch}{1.0}
\end{minipage}
\begin{minipage}[t]{0.5\linewidth}
\raggedleft
\begin{tabular}{|l|l|}
\hline
type     & distinction among \\
\hline
de re    & existing things \\
de signo & percieved signals \\
de dicto & interpreted symbols \\
\hline
\end{tabular}
\end{minipage}
\caption{Types of data, based on Floridi}
\label{tab:floridisdatatypes}
\end{table}


\subsection{Philosophy of Technology and Design}

Another path to advance philosophy of data, metadata, and digital documents,
can be found in philosophy of technology and design. Following the view of
data as bits or as other symbols, data is not observed, but created. More
precisely it is intentionally created and shaped for automatic processing
by technical systems. This property differentiates data from natural language.
In philosophy the discussion of technical system is still quite new.
Apart from brief statements in ancient literature, philosophy of technology 
has emerged in the last two centuries. After a long domination of methaphysical
and ethical questions, some philosophers turn from the discussion of the impact
of technology on society, to analytical philosophy of technology, which is more
concerned with technology itself \cite{Franssen2009}. A central term in this
field is the \Term{design process} as intentional practice to create and control
technology. Design is applied not to find out how the world is, as in science,
but to bring the world closer to how it should be. Technology always serves, or
is intended to serve, a particular purpose by executing a specific function.
By this, technology is normative, so philosophy of technology is directly 
connected to philosophy of norms \cite{Vaesen2008}.  
The products of technological design processes are called \Term[artifact]{artifacts}.
Artificats are not limited to physical objects, but they can also be repeatable
operations, for instance computer programms. Most works in philosophy of 
technology, however focus on physical artifacts or on the implication of
artifacts to the physical world.\footnote{An exception is the discussion
of virtual reality, which was popular for some years. 
% Other topics, such as artificial intelligence and transhumanism
% impact on human beings instead of finding out
}
A philosophical analysis of the design and function of data as technological
artifacts has yet to be written.

%if we want to know how data is structured and described.
%
% member of this still new subdiscipline can be found at the former 
% research project "The Dual nature of Artefacts" and in the publications
% and 
% and 
% "Philosophical Foundations of Modern Technology"
%
% http://www.dualnature.tudelft.nl/index.htm
% 
% \cite{DeRidder2007}
% Simon Stevin Series in the Philosophy of Technology
% http://www.bauer.uh.edu/parks/fis/saraswat3.htm

