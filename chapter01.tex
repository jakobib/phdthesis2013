\chapter{Introduction}
\label{ch:introduction}
\renewcommand{\thepage}{\arabic{page}}% start with roman page numbers
\setcounter{page}{1}

\begin{quotation}%
Beginning thinkers in this area often suppose that what will
be offered to the screen reader will be merely individual stored documents,
available on line quickly, but based somehow on conventional documents
ensiling in conventional computer files.

\vspace*{1mm}

Our point of view is different.\\
\quotationsource \Person[Ted]{Nelson} \cite*{Nelson1981}:
\textit{Literary Machines}, page 1/9
\end{quotation}

\section{Motivation}

Bibliographic data conceals a sneaky complexity. At first glance it is all
quite familiar and simple: there is an author, a title, and a date of
publication. Not by chance computer science publications frequently exemplify
data by bibliographic data and metadata by library catalogs. On closer
inspection everything falls apart: how about multiple authors, editors,
publishers, and translators? What if authors are unknown or known under
different names? What about subtitles and abbreviations? Which date does one
specify in which detail and when does it change? Library science has elaborated
detailed cataloging rules to answer these questions. But bibliographic data is
not created, stored, modified, and used solely by skilled librarians --- even
they do not commit to a single schema. Moreover the subject of cataloging is
changing.  Although \Person[Ted]{Nelson}'s vision of a purely digital ecosystem
of interconnected documents has not become reality yet, more and more
publications appear in digital form. Traditional concepts such as `document',
`page', `edition', and `copy' blur or change meaning --- the continuing
popularity of print-oriented techniques like the \tacro{portable document
format}{PDF} is only a sign of reaction to this process.

In library and information science the description of physical documents is
still relevant but it has largely been solved as research topic. The topic of
this thesis is the description of \emph{digital} documents which are given as
data.  While there is an increasing research interest in (digital) documents
--- see \textcite{Buckland1998}, \textcite{Pedauque2006}, and
\textcite{Skare2007} for approaches --- the nature of these documents as data
has received less attention. Digital documents and all digital content share
two crucial and basic properties, that have long been neglected in library
science: bits can freely be copied and rearranged.  Despite their nature as
artifacts of communication, physical documents are still considered as stable,
distinguishable, and atomic entities. But given a digital document one can
create any number of identical copies, indistinguishable from the original.
With same ease one can create modifications, that may or may not constitute
documents in their own right. These properties of digital documents complicate
their bibliographic description by metadata. Despite the success of full text
search for simple retrieval, the importance of metadata for digital content is
even higher than for physical objects: if metadata is used to differentiate
digital copies from each other, it does not only describe but also constitute
the document. Digital documents are also much more likely processed,
transformed, and aggregated than physical documents. In doing so, metadata is
needed to state when two different strings of bits represent the same document.
Lacking a physical structure of pages, we also require metadata to structure
documents into parts.

As documents become digital, so does metadata. To understand the nature of
bibliographic data, we must first understand the nature of data. If it is
`something given' (as indicated by the Latin origin \emph{datum}) where
does the act of giving originate? If data describes other data, what properties
does it refer to? We will approach this questions by analysis of existing
methods of structuring and describing data. The results apply to both data and
metadata.  Later it will be shown what constitutes the relationship between
data and metadata and how the results apply to metadata about digital documents
in particular. A clear distinction between data and metadata is difficult for
several reasons. Sometimes it is not even clear whether metadata is added as
description about a piece of data or whether it is part of it. Can citations be
considered as metadata only if extracted from a document? Are they metadata
about the citing document, the cited document, or both? We may answer this
question by metadata about metadata, but does this lead to an infinite chain of
descriptions?  Furthermore not only metadata but also most data is about
something. Its referent may be no document and not digital, but it is not
directly accessible: in digital environments the association between data and
its referent is always constructed by a document because non-documents, such as
people, places, and ideas, are not directly accessible in the digital realm.
The affinity between metadata and data has an effect on every new system and
method to structure data. For instance it has been mentioned from the beginning
of the \term{Semantic Web} \cite{TBL1997,Guha1997,Lassila1999} although it took
some time to shift the focus from information about documents
(`\term{information resource}s') to information about any objects
(`\term{non-information resource}s'). Meanwhile concrete forms of data are
undervalued in favor of a common data language such as the \tacro{Resource
Description Framework}{RDF}. However, a look at existing data shows that there
is not one common data language but a multiplicity of formats, languages,
systems, and structures. Data and metadata is structured in many forms, e.g.
file systems, databases, markup, formats, encodings, schemas, and queries. It
is unlikely that this plurality will be replaced by one type of data only.

This thesis will look into this variety without proposing one method of data
structuring as superior to the other. Instead I want to find common patterns as
frequent strategies that occur over and over again in data and metadata. Just
like linguistic analysis of natural language reveals insights to (social)
reality, a deeper understanding of structures in data and metadata can give
insight to structures of the world, that is reflected in data. Grammar,
dialects, rhetoric figures, and other patterns that shape natural language are
deeply studied in linguistics. A similar approach to data, which could be
called `data linguistics', is still to be founded. A deeper understanding of
data patterns is crucial especially for libraries and archival institutions.
Future librarians and archivists will likely be confronted with more and more
digital documents that have been structured and described by outdated methods.
Knowledge of common patterns in data can help when \term{digital preservation}
has failed, by application of what could be called `data archaeology'. It could
be argued that there is no need for data patterns, because concrete data
structures and models already implement and define data much more precisely.
Yet existing approaches are not enough, because they each focus to one specific
formalization method.  This practical limitation blocks the view to more
general data patterns, independent from a particular encoding, and it conceals
blind spots and weaknesses of a chosen formalism. Even a perfect theoretical
system of data, metadata and digital documents may not suffice. In practice
data is often far less organized than it was meant to be. Standards are
misinterpreted or ignored.  Documentation is sketchy. Markup and formats are
unknown or broken. Eventually documents turn out to be inherently as fuzzy as
the reality that they deal about.  At least digital libraries cannot just
reject data if it lacks appropriate descriptions, so they must recognize their
complexity and uncertainty. To understand and reveal concealed structures in
data, we must not only know the techniques that have been applied to it, but
also the patterns that underlie and motivated the application of specific
technologies. This thesis will hopefully provide at least some basic guidance
for this challenge.

\pagebreak
\section{Background}
\label{sec:background}

\begin{quotation}%
We do not, it seems, have a very clear and commonly\\
agreed upon set of notions about data.\\
\quotationsource \Person[George]{Mealy} \cite*{Mealy1967}:
\textit{Another look at data}
\end{quotation}

\noindent The main topic of this thesis is the structure and description of
data in digital documents. The concept of data is relevant to many disciplines
with various meanings: a summary of different philosophies of data is given by
\textcite{BallsunStanton2010,BallsunStanton2012} with data as the product of
objective, reproducible measurements (``data as hard numbers''), data as
product of any recorded observations (``data as observations''), and data as
processable encodings of information and knowledge (``data as bits'').  This
research commits to the third understanding of data, which is found both in
computer science and library and information science.  Without committing to a
specific definition of information and knowledge, we assume that data is given
as processable encoding of something, at least as a sequence of bits. This
definition is compatible with notions of data in some disciplines that provide
tools and related works for data research. Chapter~\ref{ch:foundations} gives
an overview of basic concepts and foundations from mathematics
(section~\ref{sec:mathematics}), computer science
(section~\ref{sec:informatics}), library and information science
(section~\ref{sec:lis}), philosophy (section~\ref{sec:philosophy}), semiotics
(section~\ref{sec:semiotics}), and pattern theory (\ref{sec:patterntheory}).
This thesis can best be located between library and information science on the
one hand and computer science on the other. Both disciplines do not deal with
data as primary topic but they prefer the term \Term{information} which data is
related to as secondary form. The scope of library and information science
includes the description of documents with data as one aspect of digital
documents. Computer science neither deals with data as such but with
computation and the implementation of automatic processes. This involves data,
but data was never the central object of research as suggested by
\textcite{Naur1966}. Over the past years there has been an increasing interest
in data motivated by the growing amount of \Term{Open Data} and tools for data
analysis. This has brought up ideas of ``data science'' and ``data journalism''
\cite{Bradshaw2011}. However, both commit to the philosophies of data as hard
numbers or data as observations as they deal with aggregating, filtering, and
visualizing large sets of data, based on statistical methods of data analysis.
Such analyses require a basic understanding of data as prerequisite but they do
not make it to their primary object of investigation.  The main concern of data
science is ``big data'', that is `'when the size of the data itself becomes
part of the problem'' \cite{Loukides2010}. In contrast, the problem of this
research is the inherent complexity of data, which exists independent from its
size. It is my aim to show how data is actually structured and described,
independent from particular application for which, and independent of
particular technologies in which data is processed.

For the most part, the research question is focused on digital documents as
instances of data. The \Term{document} is a core concept of library and
information science (see section \ref{sec:lis}).  Nevertheless, there exists no
commonly agreed upon definition, even within the discipline. As described by
\textcite{Buckland1997, Buckland1998b} the nature of (digital) documents can
better be defined in terms of function rather than format: whatever functions
as a document can be a document. The document must only be usable as recorded
``evidence in support of a fact'' \cite{Briet1951}.\footnote{Translation from
French by \textcite{Buckland1997}.  In practice a fact that is supported by a
document can be any statement, whether valid and true or not.} A digital
document, in short, can be any data object that eventually exists as sequence
of bits. Such data objects are often referred to as `information'. However
\Person[Ted]{Nelson}~(\citeyear[300]{Nelson2010}) is right as he writes in
response to a misleading summary of his hypertext system \term{Xanadu} by
\Person[Tim]{Berners-Lee}: ``not `all the world's information', but all the
world's \emph{documents}. The concept of `information' is arguable, documents
much less so.''\footnote{\label{fn:nsl}The quote from \textcite{BernersLee1999}
that Nelson refers to is: ``Ted described a futuristic project, \term{Xanadu},
in which all the world's information could be published in hypertext.''} An
important distinction between digital documents, that are subject to
bibliographic description, and general data objects or information, that are
subject to general data management in business, is the stability of documents.
While business databases are designed to cover the \emph{current} state,
bibliographic data is designed to cover what \emph{has been} published or
recorded. Description and interpretation of a digital document may change, be
extended, reduced, or turn out to be wrong. Still there is the assumption of
facts, which a document is evidence for, even if both the facts and the
documents can be expressed in several ways.  Business data in contrast
describes facts that change in time: products are created and sold, people are
hired and fired, etc. Most information systems cover both types of data: static
data, that is not changed, and dynamic data, that may change.  For instance a
library system holds description of publications: these publications are
documents which are not changed after they have been published. At the same
time the system holds descriptions of dynamic holdings, which are bought, lend,
and sort out.  Dynamic data can be transformed into static data by just
`freezing' it -- this process of archiving or preservation is one major task of
library institutions. This `frozen data' is what constitutes a digital
document. This finding, however, does not answer which data constitutes a
document and how one determines its relevant parts. The definition of a
document is either passed on to the eye of the beholder or to the level of
metadata about documents. The former cannot be automatized, and the latter
forms a digital document on its own. As there is no obvious distinction between
data and metadata, metadata only shifts the problem of document identity to
another level. Nevertheless metadata provides useful methods to tackle the
nature of digital documents in particular and data in general. To further
identify digital documents, we need to reveal structures in data and metadata,
which will be described by data patterns.

\pagebreak
\section{Method and scope}
\label{sec:method}

My research method is based on a phenomenological description of existing
methods that structure and describe data. The phenomenological method views
data as social artifacts, that cannot be described from an absolute, objective
point of view. Instead occurrences of data are studied as '''phenomena`:
appearances of things, or things as they appear in our experience``
\cite{Smith2009}.  Phenomenology as philosophical discipline has its origins in
the thinking of \person[Edmund]{Husserl} (\citeyear{Husserl1931,Husserl1986}),
followed by writings of \person[Martin]{Heidegger},
\person[Maurice]{Merleau-Ponty}, \person[Jean-Paul]{Sartre} and others.
According to \textcite[681ff.]{Spiegelberg1982} a phenomenological
investigation can be laid out in three steps: phenomenological intuiting,
phenomenological analyzing, and phenomenological describing. First, the
phenomenon must be experienced ``without becoming absorbed in it to the point
of no longer looking critically''. Second, it is examined in all of its aspects
without adhering to possibly known concepts and categories. This step
``trace[s] the elements and the structure of the phenomena obtained by
intuiting. It does not in any sense demand dissecting them into separate parts.
It comprises the distinguishing of the constituents of the phenomena as well as
the exploration of their relations to and connection with adjacent phenomena.''
Finally, the phenomenological description ``forces us to concentrate on the
central and decisive characteristics of the phenomenon and to abstract from its
accidentals''. The description should reveal the \Term{essence} of a phenomenon
and give a ``reliable guide to the listener’s own actual or potential
experience of the phenomena.''

% Another overview of the phenomenological method:
% \textcite{Schmicking2009} \ldots

The phenomenon investigated in this research is the way digital data is
structured and described. A detailed analysis of this phenomenon is given in
chapter~\ref{ch:methods}.  Chapter~\ref{ch:findings} summarizes general
constituents and provides a typology to talk about data. The essential
description of data structuring is finally provided with
chapter~\ref{ch:patterns} in form of a pattern language, as explained in
section~\ref{sec:patterntheory}.  As far as I know, a combination of
phenomenological method and pattern theory has not explicitly been practiced
before.\footnote{At least a literature search for terms like ``phenomenological method''
or ''Husserl'' combined with ``pattern theory'' and ``Alexander'' led to no
results. Only \textcite{Palmer2009} follows a related approach with a
phenomenological analysis of meta-systems and systems engineering -- among 
them some patterns and schemas.}

% TODO: ask Palmer if this is cited right (and proofread this chapter)
%
% see http://webspace.ship.edu/cgboer/phenandexist.html
% see http://mailer.fsu.edu/~bcompton/garnet-bwc9865/Dissertation.htm (3.1)
%
% In short, a phenomenon is analyzed ``purely'' by describing it from all
% possible angles without any theoretical or experimental presuppositions as
% starting point.

The research method can be justified by limitations of existing approaches.
These are either theoretical, as they normatively describe how metadata
\emph{should} be structured, or empirical but limited on statistics (\term{data
mining}) and automatic methods (\term{machine learning}). Both are limited in
their scope. Normative data descriptions do not necessarily reflect existing
data, because norms are often (mis)interpreted, ignored, and changed. In
practice, data is shaped by both explicit and implicit structures. For instance
every document in the \tacro{extensible markup language}{XML} is an ordered
tree, but the nesting and order of elements might be chosen intentionally or in
an arbitrary manner, just because there has to be some order. Data schemas,
like those expressed in \tacro{XML Schema}{XSD}, rarely cover all aspects, so
they are extended by conventions like cataloging rules and profiles, which may
only exist as conceptions. Explicit and implicit structures are intertwined on
multiple levels, and both structure and describe data.

Data mining and \term{machine learning}, on the other hand, can only recognize known
structures at one level of description, but they cannot automatically
detect and interpret unexpected kinds of structures. In a nutshell, one cannot
find out whether and how different things relate to each other by treating them
all as equal.  Quantitative methods only show patterns within the constraints
of a fixed format. For instance \term{data mining} can find co-occurrences in
data fields from a large number of records, but first one must identify what
constitutes a field and a record.  These entities are examples of general data
patterns, which we are looking for.

The goal of this thesis is not yet another, unified `\"uber-model', but
``another look at data'' as \textcite{Mealy1967} titled his early work on data
theory. Since then, information technology has created a plethora of different
models, formats, languages, and methods to structure data.  It is disputable
whether there will ever be a final unification in addition to the universal
binary code. Data integration, migration, and mapping are still useful and
worth to investigate in concrete domains, but they can only provide partial
solutions. The goal of this thesis is not to unify data structuring methods, but
to analyze, relate, and describe them. To further locate the scope of this thesis,
data is assumed to be digital, stable, and finite. Aspects of transforming 
non-digital or dynamic digital material into fixed sequences of bits are not
dealt with, and details of implementation, such as performance and security, 
are only mentioned where they show how and why specific structures have evolved.

%\pagebreak
\section{Related work}
\label{sec:relatedworks}

\begin{quotation}% Newton: Auf den Schultern von Giganten Newton (1676)
If I have not seen as far as others, it is because 
there were giants standing on my shoulders.
\\\quotationsource \Person[Harold]{Abelson}
\end{quotation}

% In computing, we mostly stand\\ on each other's feet.
% \\\quotationsource \Person[Richard Wesley]{Hamming}

\noindent Both, the scope of this thesis, and its method applied to the revealing
of patterns in data description, independent from particular technologies, are
unique. Nevertheless there are several related works to build on. These works
either deal with particular technologies and parts of the problem, or they
tackle the problem of data description from different points of view and with
different methods. The following section gives a brief overview of related work
with emphasis on concrete publications and authors.  The overview begins
historically with early and foundational works, followed by analyzes of
patterns in particular domains, including metamodels and taxonomies. Some
additional works are relevant because they share theoretical or methodological
fundamentals with this thesis. In order to better find revealed aspects of data
structuring I also paid particular attention to researchers that complained
about established treatment of data and digital documents
(\person[William]{Kent}, \person[Ted]{Nelson}, \person[Peter]{Naur}, etc.).
General approaches from specific disciplines and methods (mathematics, computer
science, library and information science, semiotics, philosophy, and the study
of patterns and pattern languages) will be dealt with in
chapter~\ref{ch:foundations}.

Some data patterns before the advent of computer systems may be found in forms
and questionnaires (page \pageref{sec:forms}). Most aspects of the of history
of forms, however, still have to be written, so I skipped this topic.  The use
of \term{bit}s as most fundamental unit of data also dates back to the
pre-electronic age.\footnote{The binary number system is attributed to
\textcite{Leibniz1703} and \textcite{Boole1854}. Earlier examples of
non-numerical uses of binary systems are the African Ifa divination and the
Chinese I~Ching hexagrams.} The first foundational analyzes were created in the
late 1950s until the 1960s --- around the same time when distinctions between
data and information were introduced \cite{Gray2003} and when computer science
emerged as an independent discipline. These early works remain important also
because they are less bound to particular technological trends and paradigms,
that later emerged. For instance the ``abstract formulation of data processing
problems'' by \textcite{Young1958} contains a clear separation between
information sets (sets of possible information items belonging to the same
class, from which data is drawn), and documents (collections of related
information items). The concept of a ``hierarchy of models'' in models of data 
\cite{Suppes1962} can also help understanding general problems of data
structuring, although early notions of data tend to conform to the idea of data
as recorded observations (see page~\pageref{sec:philofdata}). Later the
interest of researchers shifted from data to concepts like `information' (or
even `knowledge'). Against this, \textcite{Naur1966,Naur1968} suggested the
term `\term{datalogy}' for ``the science of the nature and use of data''. As
described by \textcite{Sveinsdottir1988}, Naur also criticized the focus of
computer science curricula on formalization, disregarding social aspects,
psychology of programming and applications. My application of pattern analysis
instead of mathematical formalisms follows Naurs understanding of datalogy.
Another foundational discussion on data is given by \textcite{Mealy1967} and a
response by \textcite{Chapin1968}. 

Several works deal with intellectual analysis of data patterns or similar
concepts \emph{in particular domains}: \textcite{Armstrong2006} identified
common patterns in \term{Object Orientation}  by literature analysis and named
them ``quarks''.  The most common quarks, asserted as characterizing Object
Orientation in more than every second analyzed article, are inheritance (71 of
88 articles), objects (69), classes (62), encapsulation (55), methods (50),
message passing (49), polymorphism (47), and abstraction (45). Patterns in
hierarchical documents, with focus on \acro{XML} based languages, have been
analyzed by Dattolo et al. (\citeyear{Dattolo2007a}). Their basic patterns for
segmentation and extraction of structural document elements, first identified
by \textcite{Vitali2005}, are: markers (with meaning depending on position),
atoms (such as unstructured plain text), block and inline elements, records
(sets of unordered, optional and non-repeatable elements), containers (sets of
unordered, optional and repeatable elements), and tables (sequences of
homogeneous elements).  More collections of patterns can be found in
(conceptual) \term{data modeling} literature, although their primary focus is
on problems in business enterprises. The most general compilations have been
collected by \textcite{Hay1995} and by \textcite{Silverston2001}. Among other
business tasks, the former contains a brief chapter on simple document
modeling, and the latter provides template data models for common entities such
as people, organizations, products, orders, and accounting.
 
The follow-up publications by \textcite{Hay2006} and by
\textcite{Silverston2009} describe more \emph{general modeling patterns}:
Silverston's  \term{Universal Data Model} is an example of the many
\term{metamodeling} approaches one finds in conceptual data modeling
literature.  Other examples include metamodels based on mathematical notations
\cite{Keet2008a}, meta-standards \cite{OMG2009}, and hypergraph models
\cite{Boyd2005} among other approaches. Given that even normal data modeling is
not consistently applied in practice, the practical benefit of meta-modeling
seems limited, and existing works tend to ivory-tower research. Nevertheless
some general data patterns exist in metamodels in the same way as in other kind
of data models (see section \ref{sec:metamodeling} on meta-modeling). More
practical publications include review articles that summarize and compare
specific technologies of data structuring. Examples include
\textcite{Kerschberg1976} with a taxonomy of database models,
\textcite{Kent1983b} with a taxonomy of entity-relationship models, and
\textcite{Riley2010} with a broad overview of metadata formats and
technologies.

Important principles of data structuring and description are dealt with in the
works of \Person[William]{Kent} and \Person[Ted]{Nelson}. Both criticize data
structuring and description in the best sense of the term, as they show that
different methods are possible. An example is given by \textcite{Kent1988} in a
paper on the multiplicity of forms to encode a simple fact in data. In his main
work \textcite{Kent1978} deals with the relation between data and reality.
Several fundamental issues, such as normalization and identity
\cite{Kent1983,Kent2003} are topic of his later works. His analyzes are focused
on (limitations of) data and data models in traditional databases, but he
explicitly says that the topics are relevant in general. To a large degree my
work confirms and updates Kent's results, also for new data technologies.
\Person[Ted]{Nelson} thought about properties and possibilities of \emph{purely
digital} documents before the invention of elaborated file systems, word
processing and related technologies \cite{Nelson1965}. During the last decades
he kept criticizing the way we deal with digital documents --- a way that still
resembles properties of physical media \cite{Nelson1981,Nelson1986}. His vision
of \term{hypertext}, in contrast, is based on documents which (parts of) can be
referenced, cited, and reused by deep links and \term{transclusion}
\cite{Nelson1999}.  This requires knowledge about the structure of digital
documents and methods to identify particular document pieces.

% It has been said that Kent has partly proven wrong because of 
% new techniques like object oriented modeling (said where?),
% ORM, XML, and RDF. I don't think so. 


% \textcite{Buckland1997, Buckland1991} gives an introduction to the
% question what a document is from the information science point of view.
% Information as ``thing``: \cite{Buckland1991}

Related topics that partly overlap with my research include the design of
information systems \cite{Hirschheim1995}, semantic data heterogeneity
\cite{Sheth2005,Bergman2006,Pluempitiwiriyawej2000,Hayes2008}, and the creation
of data standards \cite{Meek1995,Stamper2000}. Another concept of data patterns
has been developed by \cite{Jay1995,Jay2009}, but his model is more related to
pattern matching than design patterns. The most similar works compared to my
thesis are ISO~11404 (\citeyear{ISO11404}) with a collection of language
independent \term{data type}s \cite{Meek1994a}, and the thesis by
\textcite{Honig1975} (see \textcite{Honig1978} and appendix~\ref{appendixA}).
Honig conducted a survey of data structures in 21 representative programming
languages and database management systems, resulting in a description model
with core properties, such as homogeneity, atomicity, repeatability etc.,
similar to the patterns identified in this thesis. Honig's work, apparently, has
neither been taken up nor updated so far. Both ISO~11404 and Honig's model are
compared with the final data pattern language in section~\ref{sec:evaluation}.

% Jay: ``The pattern calculus is indeed related to shape theory.  Patterns are
% a means of describing shapes or, more generally perhaps, structures''

