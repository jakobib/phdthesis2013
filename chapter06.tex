\chapter{Conclusions}
\label{ch:conclusion}

\section{Summary and results}
\label{sec:summary}

Many methods, technologies, standards, and languages exist to structure and
describe data. The aim of this thesis is to find common features in these
methods to determine how data is actually structured and described. The study
is motivated by a growing number of purely digital documents and metadata,
which both eventually exist as sequence of bits. In contrast to existing
approaches, that commit to notions of data as recorded observations and facts,
this thesis analyzes data as signs, communicated in form of digital documents.
The document approach is rooted in library and information science as
documentation science. In this discipline digital documents and metadata are
primarily given as stable artifacts instead of processable information like in
computer science. The notion of data as documents, as applied in this thesis,
excludes statistical methods of data analysis in favour of intellectual data
analysis. The study assumes that all data is implicitly and explicitly shaped
by a process of data modeling, which is always grounded in the mind of a human
being (see figure~\ref{fig:simplifieddatamodeling} and its unpackaged version
in section~\ref{sec:datamodeling}, figure~\ref{fig:datamodeling}). The study
also denies a clear distinction between data and metadata because metadata is
both a digital document and used to structure and describe digital documents:
one's data is the other's metadata and one's metadata is the other's document.
Such relations in data, however, are not purely arbitrary but based on
conventions that have been analyzed in this thesis.

\begin{figure}[b]
\centering
\begin{tikzpicture}[orm]
\matrix[column sep=6mm] {
  \node[draw,cloud,cloud puffs=10,cloud ignores aspect,cloud puff arc=90,
        minimum width=8mm,minimum height=4mm,label=mind] (mind) {};
&
 \node[ellipse,minimum width=6mm,label=model,minimum height=4mm] (model) {};
 \draw (30:2mm) -- (-90:2mm) -- (150:2mm) -- cycle;
 \draw [draw,fill=white]  (-90:2mm) circle (.6mm);
 \draw [draw,fill=white] (30:2mm) circle (.6mm);
 \draw [draw,fill=white] (150:2mm) circle (.6mm);
&
 \node[rectangle,minimum width=6mm,minimum height=4mm,label=schema] (schema) {};
 \draw (schema.north west) rectangle (schema.center);
 \draw (schema.north) rectangle (schema.east);
 \draw (schema.west) rectangle (schema.south);
&
 \node[rectangle,minimum height=4mm] (data) {\texttt{01101}\ldots}; \\
};
\node[above=0.5mm of data] {implementation};
\draw[<->,shorten >=1mm,shorten <=2mm] (mind) to (model);
\draw[<->,shorten >=2mm,shorten <=1mm] (model) to (schema);
\draw[<->,shorten >=0mm,shorten <=2mm] (schema) to (data);
\end{tikzpicture}
\caption{Simplified data modeling process}
\label{fig:simplifieddatamodeling}
\end{figure}

The plethora of existing ways to structure and describe data was analyzed by a
\term{phenomenological research method}, which is based on three steps: first,
conceptual properties of data structuring and description were collected and
experienced critically by phenomenological intuiting. As realized in
chapter~\ref{ch:foundations} and chapter~\ref{ch:methods}, data is structured
and described in different disciplines (mathematics, computer science, library
and information science, philosophy, and semiotics) and by different practices.
Examples of these practices include encodings, identifiers, markup, formats,
schemas, and models. The most common methods to structure and describe data
include data structuring languages (section~\ref{sec:dsl}) and schema languages
(section~\ref{sec:schemas}).  After this empirical part, the methods found were
grouped using phenomenological analysis without adhering to known concepts and
categories. The result of this second step was presented in
chapter~\ref{ch:findings}: the analysis resulted in six prototypes that
categorize data methods by their primary purpose
(section~\ref{sec:categorization}).  These prototypes can be used to better
grasp the actual nature of a method, independent of its originally intended
purpose:

\begin{enumerate}
 \item encodings (most of section~\ref{sec:characters})
 \item storage systems
	 (most of sections \ref{sec:filesystems} and \ref{sec:databases})
 \item identifier and query languages
	 (most of sections \ref{sec:identifiers} and \ref{sec:queries})
 \item structuring and markup languages
	 (most of sections \ref{sec:dsl} and \ref{sec:markuplanguages})
 \item schema languages: 
	 (most of section \ref{sec:schemas})
 \item conceptual models
	 (most of sections \ref{sec:modelangs} and \ref{sec:diagrams})
\end{enumerate}

The study further revealed five basic paradigms, described in
section~\ref{sec:paradigms}, each with its benefits and drawbacks. The
paradigms provide general kinds of viewing and dealing with data and they
deeply shape the way that people deal with data structuring and description:

\begin{enumerate}
\item documents and objects
\item standards and rules
\item collections, types, and sameness
\item entities and connections
\item levels of abstractions
\end{enumerate}

The third step, that is phenomenological describing, resulted in a language of
twenty fundamental patterns in data structuring and description
(chapter~\ref{ch:patterns}). The patterns show problems and solutions which
occur over and over again in data, independent from particular technologies.
This application of the pattern language approach is novel. Existing design
patterns in software engineering refer to dynamic systems instead of static
digital documents and the patterns mostly refer to one particular method of
data description. The pattern language given in this work consists of twenty
patterns, each described with its names, problems, solutions, and consequences.
Each pattern shows general strategies in data structuring and description with
its benefits, consequences, and pitfalls, and relates this strategy to other
patterns. An overview of the pattern language is given below with a
classification of the patterns (table~\ref{tab:patternclassification}) and with
a graph of pattern connections (appendix~\ref{appendixC}).  In
section~\ref{sec:evaluation} the pattern language is compared with related
works for evaluation.

% This comparison is limited because most existing analysis of
% data description, with the notion of data as sequences of bits, aim at
% particular domains and technologies
% \cite{Armstrong2006,Vitali2005,Hay1995,Silverston2001}. The most general
% publications on data structuring, ISO~11404 (\citeyear{ISO11404}) and the
% analysis model of \textcite{Honig1978}, focus on data types in programming
% languages and in database management systems. 

This thesis collected and analyzed a wide range of traditions
(chapter~\ref{ch:foundations}), methods (chapter~\ref{ch:methods}), prototypes
and paradigms (chapter~\ref{ch:findings}), and patterns
(chapter~\ref{ch:patterns}) of data structuring and description. The results
can help data modelers and programmers to find a trade-off when selecting
methods of data structuring and description for their particular application.
Patterns can also help to identify solutions that have implicitly been
implemented in data. Last but no least the result of this thesis facilitates a
better understanding of data. Applications of the results and options for
further research will be summarized in the following sections
(\ref{sec:applications} and \ref{sec:further}) before concluding with a final
reflection (section~\ref{sec:reflection}).

% TODO: short examples are given in appendix D and E

\begin{table}
\begin{enumerate}
  \item basic patterns (page~\pageref{sec:basic-patterns}ff.)
    \begin{enumerate}
      \item pure data elements
      \begin{enumerate}
        \item \pattern{label}
        \item \pattern{atomicity}
      \end{enumerate}
      \item data elements with content
      \begin{enumerate}
        \item \pattern{size}
        \item \pattern{optionality}
        \item \pattern{prohibition}
      \end{enumerate}
    \end{enumerate}
  \item combining patterns (page~\pageref{sec:combining-patterns}ff.)
    \begin{enumerate}
      \item combine multiple elements on the same level
      \begin{enumerate}
        \item \pattern{sequence}
        \item \pattern{graph}
      \end{enumerate}
      \item combine elements by subsumption
      \begin{enumerate}
        \item \pattern{container}
        \item \pattern{dependence}
        \item \pattern{embedding}
      \end{enumerate}
    \end{enumerate}
  \item relationing patterns (page~\pageref{sec:relationing-patterns}ff.)
    \begin{enumerate}
      \item primary
      \begin{enumerate}
		\item \pattern{identifier}
		\item \pattern{derivation}
      \end{enumerate}
      \item secondary
      \begin{enumerate}
        \item \pattern{encoding}
		\item \pattern{flag}
      \end{enumerate}
      \item tertiary
      \begin{enumerate}
	    \item \pattern{normalization}
		\item \pattern{schema}
      \end{enumerate}
    \end{enumerate}
  \item continuing patterns (page~\pageref{sec:continuing-patterns}ff.)
	  \begin{enumerate}[label=\roman*.]
      \item \pattern{separator}
      \item \pattern{etcetera}
      \item \pattern{garbage}
      \item \pattern{void}
    \end{enumerate}
\end{enumerate}
\caption{Full classification of patterns in data structuring}
\label{tab:patternclassification}
\end{table}

\section{Applications}
\label{sec:applications}

The results of this thesis can be applied virtually everywhere data is
intellectually used and created, including the design of automatic methods of
data processing. In particular the identified categories, paradigms, and
patterns can help to better understand existing data and to improve (the
creation of) data models. Ideally, the results will foster a general
understanding of methods to describe and structure data, independent from
specific technologies and trends, such as programming languages, software
architectures, and storage systems.  Two specific emerging domains of
application will be described below with data archaeology
(section~\ref{sec:dataarchaeology}) and data literacy
(section~\ref{sec:datalit}).

\subsection{Data archaeology}
\label{sec:dataarchaeology}

The domain of \Term{data archaeology} is recovery of digital data in unknown or
obsolete formats. This activity is closely related to \term{data recovery},
which focuses on recovery of data from damaged media and \term{file system}s.
Data archaeology includes all methods of interpretation that follow after data
recovery.  Just like archaeology exposes layers and artifacts by excavation and
remote sensing, data archaeology can use many methods to uncover structures in
data. The most related existing discipline to data archaeology is \term{digital
forensics}.  Digital forensics has a more specific scope and its application to
more complex and heterogeneous methods of data structuring, e.g.  databases, is
in an early stage of development \cite{Olivier2009}.

The term data archaeology first appeared in 1992 in the \term{Global
Oceanographic Data Archaeology and Rescue Project}. The goal of this project
was to collect, digitize, and consolidate historical data on temperature,
chlorophyll, and plankton of the oceans \cite{GODAR2007}.  To prevent the need
of data archaeology, \Term{digital preservation} or \Term{long-term
preservation} has been established as important field in library and
information science and archival science. Digital preservation is a set of
activities aimed towards ensuring access to digital materials over time
\cite{Caplan2008}. This includes creation of descriptive metadata, protection
from change, and ensuring that a given digital publication can be read in its
original form. Two strategies are followed to manage the variety and change of
digital formats: emulation of obsolete software needed to read the data, and
conversion of data to newer formats and systems. Both ways are complex and
require constant attention.  Moreover you can only describe, emulate, and
migrate what you currently know --- but from a historical view, relevant
aspects may emerge only after years and decades.

That said, data archaeology as retrospective analysis of incompletely defined
data will gain importance. The paradigms and patterns found in this thesis will
help \term{intellectual data analysis}, which is needed to underpin and
interpret algorithmic data analysis.  Algorithmic data analysis with
\term{data mining}, \term{knowledge discovery}, and related applied sciences
provides useful tools to discover detailed views on data, but they cannot
reveal its meaning as part of social practice. For this reason it is important
to locate data archaeology in the (digital) humanities\footnote{See
\textcite{Svensson2010} for a discussion of the scope and definition of
\term{digital humanities}.} as meaningful data is always a product of human
action.  It can therefore only be studied involving the cultural context of its
creation and usage. 

As \Person[Steve]{Hoberman} points out in the third edition of \textcite[p.
63]{Kent2012}, data archaeology is also an act of \term{reverse-engineering}:
``Just as an archaeologist must try to find out what this piece of clay that was
buried under the sand for thousand of years was used for, so must we try to
figure out what these [data] fields were used for when no or little
documentation or knowledgeable people resources exist.'' The data categories,
paradigms, and patterns identified in this thesis can help to detect intended
shape and purpose of such buried data elements.


\subsection{Data literacy}
\label{sec:datalit}

The term \Term{data literacy} has gained popularity in recent years to describe
the increasing need for reading and writing data, especially among researchers.
The focus of data literacy is similar to the needs of ``data science'' and
``data journalism'' \cite{Bradshaw2011} which mainly include capabilities to
aggregate, filter and visualize large sets of data with statistical methods of
data analysis. Definitions of data literacy refer to the knowledge ``how to
obtain and manipulate data'' \cite{Schield2004} and how to ``understand, use,
and manage science data'' \cite{Qin2010}.\footnote{\textcite{Qin2010} refer to
\emph{scientific} or \emph{science} data literacy with the ability of
``collecting, processing, managing, evaluating, and using data for scientific
inquiry'' but they neither provide a separation to general data literacy nor a
definition of data.} \textcite{Carlson2011} refer to data literacy as the
capability of ``understanding what data mean, including how to read graphs and
charts appropriately, draw correct conclusions from data, and recognize when
data are being used in misleading or inappropriate ways.'' These definitions
and the majority of data literacy literature and curricula focus on numerical
data, management of scientific data sets \cite{Haendel2012}, common data
processing software, file formats, and preservation. Despite the importance of
these aspects of data, there is a lack of theory in current data literacy. In
particular, current data literacy mostly ignores the semiotic nature of data
and the conception of data as communications which are not measured or observed
but created \cite{BallsunStanton2012}. Instead the domain is committed to the
notions of data as hard numbers or data as observations and emphasises
\term{statistical literacy} to aggregate and filter large sets of data. This
thesis with its focus on data as communications can provide both, a theoretical
foundation of data literacy, and guidelines to better appraise practical method
of data structuring and description, which are already subject of current data
literacy.



\section{Further research}
\label{sec:further}

The results of this thesis should not be taken as a final product, but as a
starting point. It is natural that in a phenomenological investigation one
cannot fully experience a phenomenon in all of its aspects without getting lost
in it. The analysis of methods and systems for structuring and describing data
(chapter~\ref{ch:methods}) could be extended to additional data structuring
languages, more encodings, schema languages etc. Nevertheless it is unlikely
that new methods will change the results apart from minor corrections and
additions. In particular it may be worth to have a deeper look at the history
and practice of \term[form]{forms}, as mentioned at page~\pageref{sec:forms},
and at patterns in visual notations, such as electrical circuit diagrams (see
section~\ref{sec:diagrams} and \textcite{Tversky2011}). Specific technologies
not analyzed in more detail in this thesis include \term{zzStructure}
\cite{Nelson2004,McGuffin2004,Dattolo2009a,Pourabdollah2009b,Gutteridge2010}
and the \term{Data Format Description Language} (see page~\pageref{sec:dfdl}).
Query languages and APIs (section~\ref{sec:apis}) have also received less
attention than other methods of data structuring and description.

Especially the pattern language in chapter~\ref{ch:patterns} can be improved
continuously by further discussion and evaluation. A promising sample
application would be to categorize and analyze the data standards collected by
\textcite{Riley2010}. As noted in section~\ref{sec:evaluation}, evaluation of
the pattern language requires user studies with practitioners and experts,
which would go beyond the scope of this thesis. A possible methodology for
evaluating the pattern language has been proposed by \textcite{Petter2010}.  To
facilitate improvements and applications, the pattern language will be made
available under the CC-BY-SA license. Surely understandability and usability
can be improved by adding examples and illustrations to better convey the core
idea of each pattern.\footnote{An idea not followed in this thesis was to
depict each pattern by an icon for better recognition.}

In addition to the refinement of results of this thesis, the study can be
broadened and used as starting point for further research. The following
disciplines and activities, among others, might provide additional insights:

\begin{itemize}
  \item \Term{Information design} and \term{data visualization} aim at visual
	methods to represent and display information and data. Popular examples
	were given by \textcite{Tufte2001} and \textcite{Bertin2011}.
  \item \term{Digital forensics} already has some history and relevant practice
	in recovery of structures and descriptions from data.
  \item Mathematics may guide to applications of non-classical logic to 
	data description.
  \item In \term{data analysis}, \term{linguistic summaries of data} can be 
	created based on \term{fuzzy set theory}. These summaries provide natural 
	language statements, that capture the main characteristics of data sets 
	\cite{Yager1981,Lietard2008}.

\end{itemize}

Last but not least, the semiotic background of data could be elaborated in more
detail. At best, this thesis provides a `semiology of data' similar to the
semiology of graphics by \textcite{Bertin1967,Bertin2011}. Expanding the notion
of data as sign to data as language, this thesis might also be placed in a new
discipline called \Term{data linguistics}.  Several linguistic subfields exist,
each concerned with particular aspects of human language. For instance
anthropological linguistics and sociolinguistics study the relation between
language and society, and historical linguistics studies the history and
evolution of languages. Although digital documents are used for communication,
there is no branch of linguistics dedicated to the study data as language.

% comparative linguistics,
%comparative data linguistics (eg compare data models in different modeling
%languages or data format in different serialization languages or different
%storage formats)\ldots

% The notion of data as communications, applied in this thesis, allows to analyze
% data as signs and systems of data as languages, which can be studied by
% \Term{data semiotics} and \Term{data linguistics}. Despite the current lack of
% these disciplines as organized fields of study, one can identify several fields
% that that combine data and linguistics from different viewpoints.

% research on data linguistics, as well as the history of library catalogs.
% See http://www.buymyfonts.com/data/lectures.html and Peter Beckers
% Forschungsvorhaben Benedikt Burkard: Eine kleine Geschichte des Formulars In:
% Das Archiv 1/2010 (S. 6-13); 

% Data modeling is an intellectual activity, mostly applied as top-down design
% process. In my thesis I `put data modeling back on its feet' by analysing the
% actual forms of data.

\section{Final reflection}
\label{sec:reflection}

``We do not, it seems, have a very clear and commonly agreed upon set of
notions about data'' --- since \Person[George]{Mealy} wrote this in
\citeyear{Mealy1967} the world of data processing has changed a lot. Many
technologies and models have been proposed and applied, but the basic problem
of data modeling remains. As demonstrated by \Person[William]{Kent} in his
classic ``Data and Reality'' (\citeyear{Kent1978}), the problem is independent
from technology and it cannot be solved finally. Given the growing importance
of data and digital documents, the lack of current research about foundations
of data is surprising. It looks that since the 1980s, when computers became
mainstream, the concept of data has been accepted as given. Attention of
research is either on efficient implementations with practical value in limited
domains, or on sophisticated abstract models, little connected to data practice
with its plurality of formats and interpretations. A naive belief in progress
is visible in hype cycles around technologies and models such as \acro{ERM},
\term{Object Orientation}, \acro{XML} and \acro{RDF}. Despite the usefulness of
these methods, they do not reflect a simple progression of improvements.  As
\Person[Ted]{Nelson} (\citeyear{Nelson2012}) keeps on stressing, ``the computer
world deals with, imaginary, arbitrary, made-up stuff, that was all made up by
somebody''. Eventually, all data is created by human beings for human beings.
For this reason data is no simple expression of information or even knowledge,
but a social artifact, based on convention.  This social artifact is called a
\term{document}. Nelson talks about documents where \Person[Tim]{Berners-Lee}
and others talk about information.\footnote{Nelson explicitly coined the term
	``docuverse''. See also \textcite[300]{Nelson2010} and
	footnote~\ref{fn:nsl} at page~\pageref{fn:nsl} for a comparision. However
	both, Nelson and Berners-Lee, do not talk about totally different things as
one can show with the paradigm of documents and objects
(section~\ref{sec:docobj}).} The concept of this document, which is independent
from its physical form, can be traced back to founders of library and
information science, such as \textcite{Bush1945}, \textcite{Otlet1934},
\person[Wilhelm]{Ostwald} \cite{Hapke1999}, \person[Emanuel]{Goldberg}
\cite{Buckland2006}, and \textcite{Briet1951}. Therefore the phenomenon
investigated in this thesis, the way digital data is structured and described,
turns out to be inseparable from the nature of digital documents and metadata
in general. To understand the latter, it is necessary to understand data,
independent from technologies.

To conclude with two of the giants, whose shoulders this thesis is built on,
``it's possible to argue that this book hasn't accomplished much''
\cite[351]{Gamma1994}: this thesis does not present a new and better method to
structure and describe data. The contribution, however, is more important than
yet another data language. The prototypes, paradigms, and patterns, provide
``another look at data'' \cite{Mealy1967} by revealing unspelled assumptions
that deeply shape how data is and will be structured and described in practice.

